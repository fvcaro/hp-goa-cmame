\usepackage[main=english]{babel}
\usepackage{amsmath,amsthm,mathrsfs,amssymb,amsfonts,dsfont,nicefrac,stmaryrd,yhmath,mathscinet}
\usepackage{mathtools}
%\usepackage{accents}
%\mathtoolsset{showonlyrefs=true}
\usepackage{pgfplotstable}
\usepackage{pgfplots}
\usepackage{caption}
\usepackage{tikz}
%\pgfplotsset{compat=1.15}
%\pgfplotsset{compat=1.13}
\pgfplotsset{compat=newest}
\usepackage{pgfmath,pgffor}
%\usepackage{ifthen}
\usepackage{xifthen}
\usepackage{xstring}
\usepackage{calc}
\usepackage{excludeonly}
\usepackage{multirow}
\usepackage{floatrow}
\floatplacement{figure}{!htbp}
%\usepackage[skip=1pt]{subcaption} %Nota: si se habilita, los captions de las Figuras quedaran más cerca de las Figuras.
\usepackage{subcaption}
\usepackage{multicol}
\usepackage[inline,shortlabels]{enumitem}
%\usepackage{ulem}
\usepackage{stackengine}
\usepackage{csquotes}
\usepackage{dirtytalk}
%\usepackage[text={16cm,25cm},centering]{geometry}
\usepackage[colorinlistoftodos]{todonotes}

\usepackage[debug, hidelinks,hypertexnames=false]{hyperref} % make sure to load it last
\usepackage{cleveref}
%\usepackage{autonum} %Nota: si se habilita ninguna ecuación aparecerá enumerada
\usepackage{listings}
\usepackage{xcolor}
\usepackage[ruled,vlined]{algorithm2e}

\makeatletter
\renewcommand{\todo}[2][]{\tikzexternaldisable\@todo[#1]{#2}\tikzexternalenable}
\makeatother

\pgfdeclarelayer{background}
\pgfdeclarelayer{foreground}
\pgfsetlayers{background,main,foreground}   %% some additional layers for demo

\usetikzlibrary{positioning,patterns,shapes}
\usepgfplotslibrary{external}
\usetikzlibrary{decorations.markings,intersections,calc}
\usetikzlibrary{decorations.pathreplacing}
\usetikzlibrary{fpu}
\usepgfplotslibrary{colormaps}
\usepgfplotslibrary{colorbrewer}
\usepgfplotslibrary{polar}
\usepgfplotslibrary{ternary}
\usepgfplotslibrary{smithchart}
\usepgfplotslibrary{patchplots}
%
\usetikzlibrary{fit}
\makeatletter
\tikzset{
  fitting node/.style={
      inner sep=0pt,
      fill=none,
      draw=none,
      reset transform,
      fit={(\pgf@pathminx,\pgf@pathminy) (\pgf@pathmaxx,\pgf@pathmaxy)}
    },
  reset transform/.code={\pgftransformreset}
}

\newcommand{\Cross}{$\mathbin{\tikz [x=1ex,y=1ex,line width=.1ex, red] \draw (0,0) -- (1,1) (0,1) -- (1,0);}$}%


\makeatother

\makeatletter
\def\underbrace#1{\@ifnextchar_{\tikz@@underbrace{ \displaystyle #1}}{\tikz@@underbrace{#1}_{}}}
\def\tikz@@underbrace#1_#2{\tikz[baseline=(a.base)] {\node (a) {\(#1\)}; \draw[line cap=round,decorate,decoration={brace,amplitude=5pt}] (a.south east) -- node[below,inner sep=7pt] {\(\scriptstyle #2\)} (a.south west);}}
\makeatother

\tikzexternalize[
  mode= list and make,
  prefix=External_TikZ/
]
%\newcommand{\datafolder}{}
\tikzset{external/only named=true}

\usepackage{framed} %for the left bar


\renewenvironment{leftbar}[1][\hsize]
{%
  \def\FrameCommand
  {%
    { \vrule width 2pt}%
    \hspace{5pt}%must no space.
    \fboxsep=\FrameSep%
  }%
  \MakeFramed{\hsize#1\advance\hsize-\width\FrameRestore}%
}
{\endMakeFramed}

\newenvironment{var_for}[1]
{%
  \begin{leftbar}[0.8\textwidth]
    \noindent #1}%
    {\end{leftbar}}%
%: definition commands
\theoremstyle{remark}
\newtheorem*{remark}{Remark}
\newtheorem*{expl}{Example}

\newtheoremstyle{noparens}%
{}{}%
{}{}%
{\bfseries}%
{:}%
{ }%
{\thmnote{#3}}
%  {\thmname{#1}:~\thmnote{#3}}


\theoremstyle{noparens}
\newtheorem{defn}{Definition}

%\theoremstyle{definition}

%%%%%%%%%%%%%%%%%%%%%%%%%%%%%%%%%%%%%%%%%%%%%%%%%%%%%%%%%%%%%%%%%%%%%%%%%%%%%%

\newcommand{\R}{\mathbb{R}}
\newcommand{\C}{\mathbb{C}}
\newcommand{\Q}{\mathbb{Q}}
\newcommand{\N}{\mathbb{N}}
\newcommand{\Z}{\mathbb{Z}}
\newcommand{\K}{\mathbb{K}}
\newcommand{\Y}{\mathbb{Y}}


\renewcommand{\H}{\mathbb{H}}
\newcommand{\V}{\mathbb{V}}

\newcommand{\calB}{\mathcal{B}}
\newcommand{\calD}{\mathcal{D}}
\newcommand{\calC}{\mathcal{C}}
\newcommand{\calP}{\mathcal{P}}
\newcommand{\calS}{\mathcal{S}}
\newcommand{\calL}{\mathcal{L}}
\newcommand{\calT}{\mathcal{T}}
\newcommand{\calR}{\mathcal{R}}
\newcommand{\calU}{\mathcal{U}}
\newcommand{\calF}{\mathcal{F}}
\newcommand{\calE}{\mathcal{E}}
\newcommand{\calI}{\mathcal{I}}
\newcommand{\calO}{\mathcal{O}}
\newcommand{\lvl}{\calL}

\newcommand{\frakR}{\mathfrak{R}}

\newcommand{\tb}{\tilde{b}}
\newcommand{\dis}{\displaystyle}
\newcommand{\abs}[1]{\left|#1\right|}
\newcommand{\eps}{\varepsilon}
\newcommand{\norme}[1]{\left\|#1\right\|}
\newcommand{\norm}[1]{\left\|#1\right\|}
\newcommand{\normOP}[1]{\left\vvvert#1\right\vvvert} %needs mathabx package

\renewcommand{\leq}{\leqslant}
\renewcommand{\geq}{\geqslant}
\renewcommand{\tilde}{\widetilde}
\newcommand{\grad}{\nabla}
\newcommand{\scalaire}[2]{\left<#1\,,#2\right>}
\newcommand{\scalar}[2]{\left<#1\,,#2\right>}

%\DeclareMathOperator{\div}{div}
\DeclareMathOperator{\atan}{atan}
\DeclareMathOperator{\atanB}{atan2}

\DeclareMathOperator*{\lOngrightarrow}{\longrightarrow}
\DeclareMathOperator*{\argmin}{\textrm{arg min}}

\newcommand{\fonction}[5]{\begin{array}[t]{l|ccl}
    #1 \colon & #2 & \longrightarrow & #3 \\
              & #4 & \longmapsto     & #5
  \end{array}}
\newcommand{\fonc}[3]{#1 \colon #2 \longrightarrow #3}


\newcommand{\dof}[1]{\mathrm{Ind}_{\mathrm{basis}}(#1)}
\newcommand{\rmdof}[1]{\ifthenelse{\equal{#1}{}}{\mathrm{R}\mathrm{m}_\mathrm{dof}}{\mathrm{R}\mathrm{m}_\mathrm{dof}(#1)}}

\newcommand{\rmbasis}[1][]{\ifthenelse{\equal{#1}{}}{\mathrm{R}\mathrm{m}_\mathrm{basis}}{\mathrm{R}\mathrm{m}_\mathrm{basis}(#1)}}

%\newcommand{\rmbasis}[1][]{\mathrm{R}\mathrm{m}_\mathrm{basis}(#1)}

\newcommand{\Rm}[1]{\mathrm{Rm}(#1)}

\newcommand{\Trm}[1][]{{\calT_{#1}^{\mathrm{rm}}}}
\newcommand{\tildeTrm}{{\tilde{\calT}^{\mathrm{rm}}}}

\newcommand{\Hone}{\H^{1}}
\newcommand{\Hcurl}{\H(\mathrm{curl})}
\newcommand{\Hdiv}{\H(\mathrm{div})}
\newcommand{\Ltwo}{L^{2}}


\newlength{\plotwidth}
\newlength{\plotheight}

\newlength{\subplotwidth}
\newlength{\subplotheight}

\setlength{\plotwidth}{7cm}
\setlength{\plotheight}{7cm}

\setlength{\subplotwidth}{0.4\textwidth}
\setlength{\subplotheight}{\plotwidth}

\newcommand{\irev}[1]{\textcolor{black}{#1}}
\newcommand{\reva}{\textcolor{black}}
\newcommand{\revb}{\textcolor{black}}
\newcommand{\rev}{\textcolor{black}}
\newcommand{\irevdos}[1]{\textcolor{blue}{#1}}
\newcommand{\revados}{\textcolor{blue}}
\newcommand{\revbdos}{\textcolor{blue}}
\newcommand{\redosv}{\textcolor{blue}}
\newcommand{\dO}{d\Omega}


\pgfplotsset{surfBF/.style={surf,
      opacity=0.7,
      samples=30,
      colormap/jet,
    }}
\pgfplotsset{meshBF/.style={mesh, color=black,
      opacity=1, thick,
      samples=2,
    }}

\pgfplotsset{surfBF1/.style={surf,
      opacity=0.5,
      samples=30,
      colormap/blackwhite,
    }}

% Style to select only points from #1 to #2 (inclusive)
\pgfplotsset{select coords between index/.style 2 args={
      x filter/.code={
          \ifnum\coordindex<#1\def\pgfmathresult{}\fi
          \ifnum\coordindex>#2\def\pgfmathresult{}\fi
        }
    }}

\newcommand{\todoVD}[2][]%
{\todo[color=green!25,#1]{\footnotesize{\bf Vincent:} #2}}

\newcommand{\todoJA}[2][]%
{\todo[color=cyan!25,#1]{\footnotesize{\bf Julen:} #2}}

\newcommand{\todoDP}[2][]%
{\todo[color=blue!25,#1]{\footnotesize{\bf David:} #2}}

\newcommand{\todoFC}[2][]%
{\todo[color=red!25,#1]{\footnotesize{\bf Felipe:} #2}}

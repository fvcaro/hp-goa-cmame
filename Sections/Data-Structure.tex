% !TEX root =  ../unref_general.tex
\section{Data Structures}
\label{sec:DataStruct}

\revb{Classical adaptive schemes often refine a starting coarse mesh to obtain finer ones. While performing local $h$- or $hp$-refinements, \emph{hanging nodes} appear, and they should be constrained to guarantee the global continuity of the approximate solution. This fact often poses serious implementation difficulties (\revbdos{see, e.g.,}~\cite{rachowicz2000hp}).}

\revb{In 1971, Mote~\cite{mote1971global} proposed an alternative procedure based on the idea of refining by superposition. This approach, nowadays known as superposition techniques, maintains an initial \emph{base} discretization unmodified and subsequently overlaps one (or several) finer \emph{overlay} mesh(es). Accordingly, the initial coarse grid captures the large-scale characteristics of the solution while the overlaying mesh(es) reproduces the small-scale features. In 2015, Zander et al.~\cite{zander2015multi} took advantage of this superposition idea and proposed a data structure that enables local $hp$-mesh refinements and unrefinements while easily handling the constrained \emph{hanging nodes} that naturally appear during local $h$-refinements (\revbdos{see, e.g.,}~\cite{demkowicz2007computing,solin2003higher}). }

\revb{Following the data structures introduced in~\cite{zander2015multi}, we impose a massive number of Dirichlet nodes throughout the overlay mesh(es), thus ensuring the continuity of the solution by construction. Basically, in the overlay meshes, we only add globally continuous \emph{basis functions} (see \Cref{fig:multilevel1d}) rather than possibly discontinuous \emph{shape functions} (see~\cite{zander2015multi, darrigrand2020painless}). That leads to a rather simple implementation where imposing the one-irregularity rule~\cite{demkowicz2002fully} is unnecessary. In addition, to guarantee the linear independence of the basis functions, high-order basis functions are only activated on those elements with no further refinements in $h$ (see \Cref{fig:multilevel1d}). \revbdos{Such elements without further refinements may be encountered even in the initial level of the mesh in the case of unrefined elements.} In particular, when performing an $h$-refinement, high-order basis functions are \revbdos{transferred to the children}. For further details, we refer the reader to~\cite{zander2017multi}.}










%\revb{Classical refinement schemes often refine a starting coarse mesh to obtain finer ones. In 1971, Mote~\cite{mote1971global} proposed an alternative procedure based on the idea of refining by superposition. This approach, nowadays known as superposition techniques, maintains an initial \emph{base} discretization unmodified and subsequently overlaps one (or several) finer \emph{overlay} mesh(es). Accordingly, the initial coarse grid captures the large-scale characteristics of the solution while the overlaying mesh(es) reproduces the small-scale features. \reva{Thus}, the numerical accuracy is improved.}
%
%\revb{In 2015, Zander et al.\cite{zander2015multi} took advantage of this superposition idea and proposed a data structure that permits high order $hp$-mesh refinements and unrefinements while avoiding dealing with the compatibility requirements of the shape functions and the \emph{irregularities} generated by arbitrary-level, high-order \emph{hanging nodes}. }
% 
%\revb{Imposing a massive number of Dirichlet nodes throughout the overlay meshes ensures the continuity of the solution and avoids hanging nodes by construction. Together with the use of hierarchical basis functions in $h$, these facts allow for: (a) a simple implementation that naturally handles hanging nodes and (b) the construction of hierarchical $hp$-FEM meshes without the limitations associated with mesh irregularities. Finally, to guarantee the linear independence of the basis functions, these data structures only activate higher-order degrees of freedom at elements with no further $h$-refinements. Accordingly, they are removed from the respective meshes on all the lower levels if they previously existed. For additional details, we refer the reader to~\cite{zander2017multi}.}
%
%\Cref{fig:multilevel1d} illustrates a sequence of local $hp$-refinements over a 1D mesh: the \emph{base} (level 0) $hp$-mesh is overlapped with two overlay meshes. The white circles stand for the homogenous Dirichlet nodes and the black ones for the active nodes. \revb{Notice that this example includes high-order degrees of freedom at all levels as long they are not further $h$-refine.}.

\begin{figure}
\centering
\begin{tikzpicture}[x=4cm,y=2cm,decoration={markings,% switch on markings mark=% actually add a mark
     mark=at position 0 with{\draw (0pt,-2pt) -- (0pt,2pt);},
     mark=at position 1 with{\draw (0pt,-2pt) -- (0pt,2pt);},
}
]

\node(origin1) at (0,1) {};
\node(origin2) at (1,1) {};
\node(origin3) at (2,1) {};

\node(father1) at (0,0) {};
\node(father2) at (0.25,0) {};
\node(father3) at (0.5,0) {};
\node(father4) at (0.75,0) {};
\node(father5) at (1,0) {};
\node(father6) at (1.25,0) {};
\node(father7) at (1.5,0) {};
\node(father8) at (1.75,0) {};
\node(father9) at (2,0) {};
\node(father10) at (2.25,0) {};
\node(father11) at (2.5,0) {};
\node(father12) at (2.75,0) {};
\node(father13) at (3,0) {};

\node(son1) at (0,-1) {};
\node(son1a) at (0.125,-1) {};
\node(son1b) at (0.25,-1) {};
\node(son1c) at (0.375,-1) {};
\node(son2)  at (0.5,-1) {};
\node(son3)  at (0.75,-1) {};
\node(son4)  at (1,-1) {};
\node(son10)  at (2,-1) {};
\node(son11) at (2.25,-1) {};
\node(son12) at (2.5,-1) {};
\node(son13) at (2.75,-1) {};
\node(son14) at (3,-1) {};

\node(grandson1) at (0,-2) {};
\node(grandson2)  at (0.25,-2) {};
\node(grandson3)  at (0.5,-2) {};
\node(grandson4)  at (0.625,-2) {};
\node(grandson5)  at (0.75,-2) {};
\node(grandson6)  at (0.875,-2) {};
\node(grandson7)  at (1,-2) {};
\node(grandson8)  at (1.25,-2) {};

\node(ggson1) at (0,-3) {};
\node(ggson2)  at (0.25,-3) {};
\node(ggson3)  at (0.5,-3) {};
\node(ggson4)  at (0.625,-3) {};
\node(ggson5)  at (0.75,-3) {};
\node(ggson6)  at (0.875,-3) {};
\node(ggson7)  at (1,-3) {};
\node(ggson8)  at (1.25,-3) {};


\node(basis0_0) at ($(father1)+(0,0.5)$){};
\node(basis0_1) at ($(father5)+(0,0.5)$){};
\node(basis0_2) at ($(father9)+(0,0.5)$){};
\node(basis0_3) at ($(father13)+(0,0.5)$){};
\node(basis1_1) at ($(son2)+(0,0.5)$){};
\node(basis1_2) at ($(son11)+(0,0.5)$){};
\node(basis1_3) at ($(son12)+(0,0.5)$){};
\node(basis1_4) at ($(son13)+(0,0.5)$){};
\node(basis2_1) at ($(grandson5)+(0,0.5)$){};
\node(basis2_2) at ($(grandson8)+(0,0.5)$){};
\node(basis2_3) at ($(grandson4)+(0,0.25)$){};
\node(basis2_4) at ($(grandson6)+(0,0.25)$){};
\node(basis3_1) at ($(ggson5)+(0,0.5)$){};
\node(basis3_2) at ($(ggson8)+(0,0.5)$){};
\node(basis3_3) at ($(ggson4)+(0,0.5)$){};
\node(basis3_4) at ($(ggson6)+(0,0.5)$){};


%%%%%% HORIZONTAL LINES
% level 1
\draw [postaction={decorate}] (father1.center) -- (father5.center) node[pos=0.5](elemfather1){};
\draw [postaction={decorate}] (father5.center) -- (father9.center) node[pos=0.5](elemfather2){};
\draw [postaction={decorate}] (father9.center) -- (father13.center) node[pos=0.5](elemfather3){};

% level 2

\draw [postaction={decorate,red}] (son1.center) -- (son2.center) node[pos=0.5](elemson1){};
\draw [postaction={decorate}] (son2.center) -- (son4.center) node[pos=0.5](elemson2){};

\draw [postaction={decorate,red}] (son10.center) -- (son12.center) node[pos=0.5](elemson31){};
\draw [postaction={decorate}] (son12.center) -- (son14.center) node[pos=0.5](elemson4){};

% level 2
\draw [postaction={decorate}] (grandson3.center) -- (grandson5.center) node[pos=0.5](elemgrandson1){};
\draw [postaction={decorate}] (grandson5.center) -- (grandson7.center) node[pos=0.5](elemgrandson2){};

% k3
%\draw [postaction={decorate}] (ggson3.center) -- (ggson5.center) node[pos=0.5](elemggson1){};
%\draw [postaction={decorate}] (ggson5.center) -- (ggson7.center) node[pos=0.5](elemggson2){};

%%%%%%%%% BOTTOM LINE %%%%%%%%%

\node(gggson1) at (0,-2.5) {};
\node(gggson2)  at (0.125,-2.5) {};
\node(gggson3)  at (0.25,-2.5) {};
\node(gggson4)  at (0.375,-2.5) {};
\node(gggson5)  at (0.5,-2.5) {};
\node(gggson6)  at (0.625,-2.5) {};
\node(gggson7)  at (0.75,-2.5) {};
\node(gggson8a)  at (0.8125,-2.5) {};
\node(gggson8)  at (0.875,-2.5) {};
\node(gggson8b)  at (0.9325,-2.5) {};
\node(gggson9)  at (1,-2.5) {};
\node(gggson10)  at (1.125,-2.5) {};
\node(gggson11)  at (1.25,-2.5) {};
\node(gggson12)  at (1.375,-2.5) {};
\node(gggson13)  at (1.5,-2.5) {};
\node(gggson14)  at (1.625,-2.5) {};
\node(gggson15)  at (1.75,-2.5) {};
\node(gggson16)  at (1.875,-2.5) {};
\node(gggson17)  at (2,-2.5) {};
\node(gggson18)  at (2.125,-2.5) {};
\node(gggson19)  at (2.25,-2.5) {};
\node(gggson20)  at (2.375,-2.5) {};
\node(gggson21)  at (2.5,-2.5) {};
\node(gggson22)  at (2.625,-2.5) {};
\node(gggson23)  at (2.75,-2.5) {};
\node(gggson24)  at (2.875,-2.5) {};
\node(gggson25)  at (3,-2.5) {};

\draw [|-] (gggson1.center) -- (gggson5.center) node[pos=0.5](){};
\draw [|-] (gggson5.center) -- (gggson7.center) node[pos=0.5](){};
\draw [|-] (gggson7.center) -- (gggson9.center) node[pos=0.5](){};
\draw [|-] (gggson9.center) -- (gggson17.center) node[pos=0.5](){};
\draw [|-] (gggson17.center) -- (gggson21.center) node[pos=0.5](){};
\draw [|-|] (gggson21.center) -- (gggson25.center) node[pos=0.5](){};


%%%%%%%%% VERTICAL LINES %%%%%%%%%

% From level 0
\draw[dotted] (father1.center) -- (gggson1.center);
\draw[dotted] (father5.center) -- (gggson9.center);
\draw[dotted] (father9.center) -- (gggson17.center);
\draw[dotted] (father13.center) -- (gggson25.center);

%\draw[loosely dotted,very thin] (father6.center) -- (gggson11.center);
%\draw[loosely dotted,very thin] (father7.center) -- (gggson13.center);
%\draw[loosely dotted,very thin] (father8.center) -- (gggson15.center);
%\draw[loosely dotted,very thin] (father11.center) -- (gggson21.center);

% From level 1
\draw[loosely dotted] (son2.center) -- (gggson5.center);
%\draw[loosely dotted] (son4.center) -- (gggson9.center);

%\draw[loosely dotted, very thin] (son1a.center) -- (gggson2.center);
%\draw[loosely dotted, very thin] (son1b.center) -- (gggson3.center);
%\draw[loosely dotted, very thin] (son1c.center) -- (gggson4.center);
%\draw[loosely dotted, very thin] (son11.center) -- (gggson19.center);
\draw[loosely dotted, very thin] (son12.center) -- (gggson21.center);
%\draw[loosely dotted, very thin] (son13.center) -- (gggson23.center);


% From level 2
\draw[loosely dotted] (grandson5.center) -- (gggson7.center);

%\draw[loosely dotted, very thin] (ggson4.center) -- (gggson6.center);
%\draw[loosely dotted, very thin] (ggson6.center) -- (gggson6.center);
%\draw[loosely dotted, very thin] (ggson6.center) -- (gggson8.center);

%\node at (gggson1)  {\Cross};
%\node at (gggson2)  {\Cross};
%\node at (gggson3)  {\Cross};
%\node at (gggson4)  {\Cross};
%\node at (gggson5)  {\Cross};
%\node at (gggson6)  {\Cross};
%\node at (gggson7)  {\Cross};
%\node at (gggson8a)  {\Cross};
%\node at (gggson8)  {\Cross};
%\node at (gggson8b)  {\Cross};
%\node at (gggson9)  {\Cross};
%\node at (gggson11)  {\Cross};
%\node at (gggson13)  {\Cross};
%\node at (gggson15)  {\Cross};
%\node at (gggson17)  {\Cross};
%%\node at (gggson19)  {\Cross};
%\node at (gggson21)  {\Cross};
%%\node at (gggson23)  {\Cross};
%\node at (gggson25)  {\Cross};

%%%%%%%%% BASIS FUNCTIONS %%%%%%%%%
%%%%%% level 0

% linear basis function

\draw[color=black, thick] (basis0_0.center) -- (father5.center);
\draw[color=black, thick] (father1.center) -- (basis0_1.center) -- (father9.center);
\draw[color=black, thick] (father5.center) -- (basis0_2.center) -- (father13.center);
\draw[color=black, thick] (basis0_3.center) -- (father9.center);

% degree 2 basis function
\draw[color=black, thick]    (father5.center) .. controls ($(elemfather2)+(-0.05,+0.15)$) and ($(elemfather2)+(0.05,0.15)$) ..  (father9.center) ;
%\draw[color=black, thick]    (father9.center) .. controls ($(elemfather3)+(-0.05,+0.15)$) and ($(elemfather3)+(0.05,0.15)$) ..  (father13.center) ;

% degree 3 basis function
\draw[color=red, thick]   (father5.center) .. controls ($(elemfather2)+(-0.01,-0.25)$) and ($(elemfather2)+(0.00,0.25)$) ..  (father9.center);

%%%%% level 1
% linear basis function
\draw[color=black, thick] (son1.center) -- (basis1_1.center) -- (son4.center);
\draw[color=red, thick] (son10.center) -- (basis1_3.center) -- (son14.center);
%\draw[color=red, thick] (son10.center) -- (basis1_2.center) -- (son12.center);
%\draw[color=red, thick] (son12.center) -- (basis1_4.center) -- (son14.center);

% degree 2 basis function
\draw[color=black, thick]    (son1.center) .. controls ($(elemson1)+(-0.05,+0.15)$) and ($(elemson1)+(0.05,0.15)$) ..  (son2.center);

% degree 3 basis function
\draw[color=red, thick]   (son1.center) .. controls ($(elemson1)+(-0.0,-0.25)$) and ($(elemson1)+(0.0,0.25)$) ..  (son2.center);


%%%%% level 2

% linear basis function
\draw[color=black, thick] (grandson3.center) -- (basis2_1.center) -- (grandson7.center);
%\draw[color=red, thick]    (grandson3.center) -- (basis2_3.center) -- (grandson5.center);
%\draw[color=red, thick]    (grandson5.center) -- (basis2_4.center) -- (grandson7.center);

% degree 2 basis function
\draw[color=red, thick]    (grandson3.center) .. controls ($(elemgrandson1)+(-0.05,+0.15)$) and ($(elemgrandson1)+(0.05,0.15)$) ..  (grandson5.center);
\draw[color=black, thick]    (grandson5.center) .. controls ($(elemgrandson2)+(-0.05,+0.15)$) and ($(elemgrandson2)+(0.05,0.15)$) ..  (grandson7.center);

% degree 3 basis function
\draw[color=red, thick]   (grandson5.center) .. controls ($(elemgrandson2)+(-0.0,-0.25)$) and ($(elemgrandson2)+(0.0,0.25)$) ..  (grandson7.center);


%%%%% k3

% linear basis function
%\draw[color=red, thick]    (ggson3.center) -- (basis3_3.center) -- (ggson5.center);
%\draw[color=red, thick]    (ggson5.center) -- (basis3_4.center) -- (ggson7.center);

%%%%%%%%% CIRCLES %%%%%%%%%

%%%%%% level 0

\node[circle,draw, fill=black, inner sep=0pt,minimum size=5pt] at ($(father1)+(0,0)$){};
%\node[circle,draw, fill=white, inner sep=0pt,minimum size=5pt] at ($(father1a)+(0,0)$){};
%\node[circle,draw, fill=white, inner sep=0pt,minimum size=5pt] at ($(father1b)+(0,0)$){};
\node[circle,draw, fill=black, inner sep=0pt,minimum size=5pt] at ($(father5)+(0,0)$){};
%\node[circle,draw, fill=black, inner sep=0pt,minimum size=5pt] at ($(father6)+(0,0)$){};
%\node[circle,draw, fill=black, inner sep=0pt,minimum size=5pt] at ($(father7)+(0,0)$){};
%\node[circle,draw, fill=black, inner sep=0pt,minimum size=5pt] at ($(father8)+(0,0)$){};
\node[circle,draw, fill=black, inner sep=0pt,minimum size=5pt] at ($(father9)+(0,0)$){};
%\node[circle,draw, fill=black, inner sep=0pt,minimum size=5pt] at ($(father11)+(0,0)$){};
\node[circle,draw, fill=black, inner sep=0pt,minimum size=5pt] at ($(father13)+(0,0)$){};

%%%%%% level 1

\node[circle,draw, fill=white, inner sep=0pt,minimum size=5pt] at ($(son1)+(0,0)$){};

%\node[circle,draw, fill=black, inner sep=0pt,minimum size=5pt] at ($(son1a)+(0,0)$){};
%\node[circle,draw, fill=black, inner sep=0pt,minimum size=5pt] at ($(son1b)+(0,0)$){};
%\node[circle,draw, fill=black, inner sep=0pt,minimum size=5pt] at ($(son1c)+(0,0)$){};
\node[circle,draw, fill=black, inner sep=0pt,minimum size=5pt] at ($(son2)+(0,0)$){};
\node[circle,draw, fill=white, inner sep=0pt,minimum size=5pt] at ($(son4)+(0,0)$){};
\node[circle,draw, fill=white, inner sep=0pt,minimum size=5pt] at ($(son10)+(0,0)$){};
%\node[circle,draw, fill=black, inner sep=0pt,minimum size=5pt] at ($(son11)+(0,0)$){};
\node[circle,draw, fill=black, inner sep=0pt,minimum size=5pt] at ($(son12)+(0,0)$){};
%\node[circle,draw, fill=black, inner sep=0pt,minimum size=5pt] at ($(son13)+(0,0)$){};
\node[circle,draw, fill=white, inner sep=0pt,minimum size=5pt] at ($(son14)+(0,0)$){};

%%%%%% level 2
\node[circle,draw, fill=white, inner sep=0pt,minimum size=5pt] at ($(grandson3)+(0,0)$){};
%\node[circle,draw, fill=black, inner sep=0pt,minimum size=5pt] at ($(grandson4)+(0,0)$){};
\node[circle,draw, fill=black, inner sep=0pt,minimum size=5pt] at ($(grandson5)+(0,0)$){};
%\node[circle,draw, fill=black, inner sep=0pt,minimum size=5pt] at ($(grandson6)+(0,0)$){};
\node[circle,draw, fill=white, inner sep=0pt,minimum size=5pt] at ($(grandson7)+(0,0)$){};

%%%%%% level 2
%\node[circle,draw, fill=white, inner sep=0pt,minimum size=5pt] at ($(ggson3)+(0,0)$){};
%\node[circle,draw, fill=black, inner sep=0pt,minimum size=5pt] at ($(ggson4)+(0,0)$){};
%\node[circle,draw, fill=white, inner sep=0pt,minimum size=5pt] at ($(ggson5)+(0,0)$){};
%\node[circle,draw, fill=black, inner sep=0pt,minimum size=5pt] at ($(ggson6)+(0,0)$){};
%\node[circle,draw, fill=white, inner sep=0pt,minimum size=5pt] at ($(ggson7)+(0,0)$){};

%%%%%%%%%%% LEGEND

%\draw[blue, very thick] at (origin) rectangle (3,2);
%\node[rectangle,draw] at (origin1) {};

\node[circle,draw, fill=white, inner sep=0pt,minimum size=5pt] at (origin1){};
\node at ($(origin1) +(0.35,0.01)$)  {Dirichlet nodes};

\node[circle,draw, fill=black, inner sep=0pt,minimum size=5pt] at (origin2){};
\node at ($(origin2) +(0.3,0.01)$)  {Active nodes};

\node[rectangle, inner sep=0pt, minimum height=1pt, minimum width=6pt, draw=red, fill=red] at (origin3)  {};
%\node at (origin3)  {\Cross};
\node at ($(origin3) +(0.4,0)$)  {Removable basis};

%%%%%%%%%%%  TEXT

%%%% left  %%%%

\node[anchor=west] at ($(father1)-(0.9,0)$) {level 0 (base)};
\node[anchor=west] at ($(son1)-(0.75,0)$) {level 1};
\node[anchor=west] at ($(grandson1)-(0.75,0)$) {level 2};
%\node[anchor=west] at ($(ggson1)-(0.75,0)$) {level 3};
\node[anchor=west] at ($(gggson1)-(0.9,0)$) {Overlapped mesh};

%%%% bottom  %%%%

\node at ($(gggson3) +(0,-0.2)$)  {$p=3$};
\node at ($(gggson6) +(0,-0.2)$)  {$p=2$};
\node at ($(gggson8) +(0,-0.2)$)  {$p=3$};
\node at ($(gggson13) +(0,-0.2)$)  {$p=3$};
\node at ($(gggson19) +(0,-0.2)$)  {$p=1$};
\node at ($(gggson23) +(0,-0.2)$)  {$p=1$};



\end{tikzpicture}
\caption{Illustration of a 1D multi-level $hp$-grid with hierarchical basis functions and Dirichlet nodes. {\em  Removable} basis functions are indicated in red.}
\label{fig:multilevel1d}
\end{figure}

  
\subsection{Removable basis functions in a multi-level $hp$-mesh}
\label{sec:removable}
In 2020, Darrigrand et al.~\cite{darrigrand2020painless} proposed an easy-to-implement $hp$-adaptive strategy for elliptic problems that exploited Zander's data structures~\cite{zander2017multi}. The main idea of this work consists of incorporating a coarsening strategy that identifies the basis functions that can be \emph{directly} removed. Hence, we define these \emph{removable} basis functions as those we can eliminate from the discretization without modifying any other basis function and preserving complete polynomial spaces. \Cref{fig:multilevel1d} shows the removable basis functions in red and non-removable basis functions in black.

For 2D and 3D problems, our current implementation defines the basis functions as tensorial products of the 1D basis functions. Additionally, we incorporate anisotropic $p$ and isotropic $h$ refinements. However, according to the recent work of Zander et al.~\cite{zander2022anisotropic}, it could be possible to extend these ideas to anisotropic $h$-refinements. To find specific details about the discretization and the properties of the genealogy tree (which are beyond the scope of this article), we refer to~\cite{darrigrand2020painless}, and for further details and the specifications about the extension to 2D and 3D data structures, we refer to~\cite{zander2017multi}. 



%\begin{figure}
%  \begin{subfigure}[]{0.45\textwidth}
%    \begin{center}
%      \centering
%      \begin{tikzpicture}[scale = 1]
%        \draw[xstep=1, ystep=1, black] (0,0) rectangle (3,3);
%      \end{tikzpicture}
%      \caption{Unrefined $2$D element}
%      \label{subfig:unref2Delement}
%    \end{center}
%  \end{subfigure}
%  \begin{subfigure}[]{0.45\textwidth}
%    \begin{center}
%      \centering
%      \begin{tikzpicture}[scale = 1]
%        \draw[xstep=1, ystep=1, black] (0,0) rectangle (3,3);
%        \draw[xstep=1.5, ystep=1.5, black] (0,0) grid (3,3);
%      \end{tikzpicture}
%      \caption{Refined $2$D element}
%      \label{subfig:ref2Delement}
%    \end{center}
%  \end{subfigure}
%  \caption{Hierarchical isotropic mesh element subdivision for $2$D cases.}
%  \label{fig:Hierarchi2D}
%\end{figure}
%
%\begin{figure}
%  \begin{subfigure}[]{0.45\textwidth}
%    \begin{center}
%      \centering
%      \begin{tikzpicture}[scale = 1]
%        %%%%%%%%%%%%%%%%%%%%%%%% Cara trasera %%%%%%%%%%%%%%%%%%%%%%%%%%%%%%%%%%%
%        \draw[gray, fill opacity=0.5, dashed] (0,0,0) -- (3,0,0); %Edge inferior punteado
%        \draw[line width=0.2mm] (3,0,0) -- (3,3,0); %Edge derecho
%        \draw[line width=0.2mm] (3,3,0) -- (0,3,0); %Edge superior
%        \draw[gray, fill opacity=0.5, dashed] (0,3,0) -- (0,0,0); %Edge izquierdo punteado
%        %%%%%%%%%%%%%%%%%%%%%%%% Cara delantera %%%%%%%%%%%%%%%%%%%%%%%%%%%%%%%%%%
%        \draw[line width=0.2mm] (0,0,3) -- (3,0,3); %Edge bajo
%        \draw[line width=0.2mm] (3,0,3) -- (3,3,3);
%        \draw[line width=0.2mm] (3,3,3) -- (0,3,3);
%        \draw[line width=0.2mm] (0,3,3) -- (0,0,3);
%        \draw[line width=0.2mm] (0,3,0) -- (0,3,3);
%        \draw[line width=0.2mm] (3,3,0) -- (3,3,3);
%        \draw[gray, fill opacity=0.5, dashed] (0,0,0.05) -- (0,0,3); %Línea entre caras
%        \draw[line width=0.2mm] (3,0,0) -- (3,0,3);
%      \end{tikzpicture}
%      \caption{Unrefined $3$D element}
%      \label{subfig:unref3Delement}
%    \end{center}
%  \end{subfigure}
%  \begin{subfigure}[]{0.45\textwidth}
%    \begin{center}
%      \centering
%      \begin{tikzpicture}[scale = 1]
%        %%%%%%%%%%%%%%%%%%%%%%%% Cara trasera %%%%%%%%%%%%%%%%%%%%%%%%%%%%%%%%%%%
%        \draw[gray, fill opacity=0.5, dashed] (0,0,0) -- (3,0,0); %Edge inferior punteado
%        \draw[line width=0.2mm] (3,0,0) -- (3,3,0); %Edge derecho
%        \draw[line width=0.2mm] (3,3,0) -- (0,3,0); %Edge superior
%        \draw[gray, fill opacity=0.5, dashed] (0,3,0) -- (0,0,0); %Edge izquierdo punteado
%        %%%%%%%%%%%%%%%%%%%%%%%% Cara delantera %%%%%%%%%%%%%%%%%%%%%%%%%%%%%%%%%%
%        \draw[line width=0.2mm] (0,0,3) -- (3,0,3); %Edge bajo
%        \draw[line width=0.2mm] (3,0,3) -- (3,3,3);
%        \draw[line width=0.2mm] (3,3,3) -- (0,3,3);
%        \draw[line width=0.2mm] (0,3,3) -- (0,0,3);
%        \draw[line width=0.2mm] (0,3,0) -- (0,3,3);
%        \draw[line width=0.2mm] (3,3,0) -- (3,3,3);
%        \draw[gray, fill opacity=0.5, dashed] (0,0,0.05) -- (0,0,3); %Línea entre caras
%        \draw[line width=0.2mm] (3,0,0) -- (3,0,3);
%        %%%%%%%%%%%%%%%%%%% Mallado: caras delantera y trasera %%%%%%%%%%%%%%%%%%%%%%%%%%%%%
%        \draw[line width=0.2mm] (0,1.5,3) -- (3,1.5,3);
%        \draw[line width=0.2mm] (1.5,0,3) -- (1.5,3,3);
%        \draw[gray, fill opacity=0.5, dashed] (0,1.5,0) -- (3,1.5,0);
%        \draw[gray, fill opacity=0.5, dashed] (1.5,0,0) -- (1.5,3,0);
%        %%%%%%%%%%%%%%%%%%%%%% Mallado: caras superior e inferior %%%%%%%%%%%%%%%%%%%%%%%%%%%
%        \draw[line width=0.2mm] (0,3,1.5) -- (3,3,1.5);
%        \draw[line width=0.2mm] (1.5,3,3) -- (1.5,3,0);
%        \draw[gray, fill opacity=0.5, dashed] (0,0,1.5) -- (3,0,1.5);
%        \draw[gray, fill opacity=0.5, dashed] (1.5,0,3) -- (1.5,0,0);
%        %%%%%%%%%%%%%%%%%%%%%% Mallado: caras laterales %%%%%%%%%%%%%%%%%%%%%%%%%%%%%%%%
%        \draw[line width=0.2mm] (3,3,1.5) -- (3,0,1.5);
%        \draw[line width=0.2mm] (3,1.5,3) -- (3,1.5,0);
%        \draw[gray, fill opacity=0.5, dashed] (0,3,1.5) -- (0,0,1.5);
%        \draw[gray, fill opacity=0.5, dashed] (0,1.5,3) -- (0,1.5,0);
%        %%%%%%%%%%%%%%%%%%%%%% Mallado: líneas interiores %%%%%%%%%%%%%%%%%%%%%%%%%%%%%%%
%        \draw[gray, fill opacity=0.5, dashed] (0,1.5,1.5) -- (3,1.5,1.5);
%        \draw[gray, fill opacity=0.5, dashed] (1.5,1.5,3) -- (1.5,1.5,0);
%        \draw[gray, fill opacity=0.5, dashed] (1.5,3,1.5) -- (1.5,0,1.5);
%        %%%%%%%%%%%%%%%%%%%%%% Algunos nodos %%%%%%%%%%%%%%%%%%%%%%%%%%%%%%%%%%%%
%      \end{tikzpicture}
%      \caption{Refined $3$D element}
%      \label{subfig:ref3Delement}
%    \end{center}
%  \end{subfigure}
%  \caption{Hierarchical isotropic mesh element subdivision for $3$D cases.}
%  \label{fig:Hierarchi3D}
%\end{figure}

%
%\begin{figure}
%  \begin{subfigure}[]{0.45\textwidth}
%    \begin{center}
%      \tikzsetnextfilename{Hier1}
%      \begin{tikzpicture}[scale=1]%,every node/.style={minimum size=1cm},on grid]
%        \begin{scope}[yshift=0]
%          \begin{axis}[view={40}{70}, ticks=none,hide axis,]
%            \addplot3[patch, patch type=rectangle,
%              patch refines=1, mesh, black, thick,
%              z filter/.code={\def\pgfmathresult{-0.0}} % change that number to translate the "mesh" (beware it may change the color mapping)
%            ] coordinates {
%                (0,0,0) (3,0,0) (3,3,0) (0,3,0)
%              };
%
%            \coordinate (a) at (0,0,0);
%            \coordinate (b) at (1.5,0,0);
%            \coordinate (c) at (1.5,1.5,1);
%            \coordinate (d) at (0,1.5,0);
%
%
%            \addplot3[
%              surfBF,
%              domain=0:3/2,domain y=3/2:3,
%            ]
%            {(-4*x*(-3+y))/9};
%            \addplot3[
%              meshBF,
%              domain=0:3/2,domain y=3/2:3,
%            ]
%            {(-4*x*(-3+y))/9};
%
%            \addplot3[
%              surfBF1,
%              domain=0:3/2,domain y=0:3/2,
%            ]
%            {(4*x*y)/9};
%            \addplot3[
%              meshBF,
%              domain=0:3/2,domain y=0:3/2,
%            ]
%            {(4*x*y)/9};
%
%            \addplot3[
%              surfBF,
%              domain=3/2:3,domain y=0:3/2,
%            ]
%            {(-4*(-3+x)*y)/9};
%            \addplot3[
%              meshBF,
%              domain=3/2:3,domain y=0:3/2,
%            ]
%            {(-4*(-3+x)*y)/9};
%
%            \addplot3[
%              surfBF,
%              domain=3/2:3,domain y=3/2:3,
%            ]
%            {(4*(-3+x)*(-3+y))/9};
%            \addplot3[
%              meshBF,
%              domain=3/2:3,domain y=3/2:3,
%            ]
%            {(4*(-3+x)*(-3+y))/9};
%
%          \end{axis}
%
%        \end{scope}
%        \begin{scope}[yshift=100]
%          \begin{axis}[view={40}{77}, ticks=none,hide axis,xmax=3,ymax=3]
%            % \addplot3[patch, patch type=rectangle,opacity=0,
%            % patch refines=1, mesh, black, thick,
%            % z filter/.code={\def\pgfmathresult{-0.0}} % change that number to translate the "mesh" (beware it may change the color mapping)
%            % ] coordinates {
%            % 	(0,0,0) (3,0,0) (3,3,0) (0,3,0)
%            % };
%            \addplot3[patch, patch type=rectangle,
%              patch refines=1, mesh, black, thick,
%              z filter/.code={\def\pgfmathresult{-0.0}} % change that number to translate the "mesh" (beware it may change the color mapping)
%            ] coordinates {
%                (0,0,0) (3/2,0,0) (3/2,3/2,0) (0,3/2,0)
%              };
%
%            \coordinate (e) at (0,0,0);
%            \coordinate (f) at (3/2,0,0);
%            \coordinate (g) at (3/2,3/2,0);
%            \coordinate (h) at (0,3/2,-3.5);
%
%            \addplot3[
%              surfBF,
%              domain=0:3/4,domain y=3/4:3/2,
%            ]
%            {(-8*x*(-3+2*y))/9};
%            \addplot3[
%              meshBF,
%              domain=0:3/4,domain y=3/4:3/2,
%            ]
%            {(-8*x*(-3+2*y))/9};
%
%            \addplot3[
%              surfBF,
%              domain=0:3/4,domain y=0:3/4,
%            ]
%            {(16*x*y)/9};
%            \addplot3[
%              meshBF,
%              domain=0:3/4,domain y=0:3/4,
%            ]
%            {(16*x*y)/9};
%
%            \addplot3[
%              surfBF,
%              domain=3/4:3/2,domain y=0:3/4,
%            ]
%            {(-8*(-3+2*x)*y)/9};
%            \addplot3[
%              meshBF,
%              domain=3/4:3/2,domain y=0:3/4,
%            ]
%            {(-8*(-3+2*x)*y)/9};
%
%            \addplot3[
%              surfBF,
%              domain=3/4:3/2,domain y=3/4:3/2,
%            ]
%            {(4*(-3+2*x)*(-3+2*y))/9};
%            \addplot3[
%              meshBF,
%              domain=3/4:3/2,domain y=3/4:3/2,
%            ]
%            {(4*(-3+2*x)*(-3+2*y))/9};
%
%
%          \end{axis}
%
%        \end{scope}
%        \draw[thick,fill opacity=1.1,dashed] (a)--(e);
%        \draw[thick,fill opacity=1.1,dashed] (b)--(f);
%        \draw[thick,fill opacity=1.1,dashed] (c)--(g);
%        \draw[thick,fill opacity=1.1,dashed] (d)--(h);
%      \end{tikzpicture}
%      \caption{$h$-refinement}
%      \label{subfig:href}
%    \end{center}
%  \end{subfigure}
%  \begin{subfigure}[]{0.45\textwidth}
%    \begin{center}
%      \tikzsetnextfilename{Hier2}
%      \begin{tikzpicture}[scale=1]%,every node/.style={minimum size=1cm},on grid]
%        \begin{scope}[yshift=0]
%          \begin{axis}[view={40}{70}, ticks=none,hide axis,]
%            \addplot3[patch, patch type=rectangle,
%              patch refines=1, mesh, black, thick,
%              z filter/.code={\def\pgfmathresult{-0.0}} % change that number to translate the "mesh" (beware it may change the color mapping)
%            ] coordinates {
%                (0,0,0) (3,0,0) (3,3,0) (0,3,0)
%              };
%
%            \coordinate (a) at (0,0,0);
%            \coordinate (b) at (1.5,0,0);
%            \coordinate (c) at (1.5,1.5,1);
%            \coordinate (d) at (0,1.5,0);
%
%
%            \addplot3[
%              surfBF,
%              domain=0:3/2,domain y=3/2:3,
%            ]
%            {(-4*x*(-3+y))/9};
%            \addplot3[
%              meshBF,
%              domain=0:3/2,domain y=3/2:3,
%            ]
%            {(-4*x*(-3+y))/9};
%
%            \addplot3[
%              surfBF,
%              domain=0:3/2,domain y=0:3/2,
%            ]
%            {(4*x*y)/9};
%            \addplot3[
%              meshBF,
%              domain=0:3/2,domain y=0:3/2,
%            ]
%            {(4*x*y)/9};
%
%            \addplot3[
%              surfBF,
%              domain=3/2:3,domain y=0:3/2,
%            ]
%            {(-4*(-3+x)*y)/9};
%            \addplot3[
%              meshBF,
%              domain=3/2:3,domain y=0:3/2,
%            ]
%            {(-4*(-3+x)*y)/9};
%
%            \addplot3[
%              surfBF,
%              domain=3/2:3,domain y=3/2:3,
%            ]
%            {(4*(-3+x)*(-3+y))/9};
%            \addplot3[
%              meshBF,
%              domain=3/2:3,domain y=3/2:3,
%            ]
%            {(4*(-3+x)*(-3+y))/9};
%
%          \end{axis}
%
%        \end{scope}
%        \begin{scope}[yshift=100]
%          \begin{axis}[view={40}{77}, ticks=none,hide axis,xmax=3,ymax=3]
%            % \addplot3[patch, patch type=rectangle,opacity=0,
%            % patch refines=1, mesh, black, thick,
%            % z filter/.code={\def\pgfmathresult{-0.0}} % change that number to translate the "mesh" (beware it may change the color mapping)
%            % ] coordinates {
%            % 	(0,0,0) (3,0,0) (3,3,0) (0,3,0)
%            % };
%            \addplot3[patch, patch type=rectangle,
%              patch refines=1, mesh, black, thick,
%              z filter/.code={\def\pgfmathresult{-0.0}} % change that number to translate the "mesh" (beware it may change the color mapping)
%            ] coordinates {
%                (0,0,0) (3/2,0,0) (3/2,3/2,0) (0,3/2,0)
%              };
%
%            \coordinate (e) at (0,0,0);
%            \coordinate (f) at (3/2,0,0);
%            \coordinate (g) at (3/2,3/2,0);
%            \coordinate (h) at (0,3/2,-3.5);
%
%            \addplot3[
%              surfBF1,
%              domain=0:3/4,domain y=3/4:3/2,
%            ]
%            {(-8*x*(-3+2*y))/9};
%            \addplot3[
%              meshBF,
%              domain=0:3/4,domain y=3/4:3/2,
%            ]
%            {(-8*x*(-3+2*y))/9};
%
%            \addplot3[
%              surfBF1,
%              domain=0:3/4,domain y=0:3/4,
%            ]
%            {(16*x*y)/9};
%            \addplot3[
%              meshBF,
%              domain=0:3/4,domain y=0:3/4,
%            ]
%            {(16*x*y)/9};
%
%            \addplot3[
%              surfBF1,
%              domain=3/4:3/2,domain y=0:3/4,
%            ]
%            {(-8*(-3+2*x)*y)/9};
%            \addplot3[
%              meshBF,
%              domain=3/4:3/2,domain y=0:3/4,
%            ]
%            {(-8*(-3+2*x)*y)/9};
%
%            \addplot3[
%              surfBF1,
%              domain=3/4:3/2,domain y=3/4:3/2,
%            ]
%            {(4*(-3+2*x)*(-3+2*y))/9};
%            \addplot3[
%              meshBF,
%              domain=3/4:3/2,domain y=3/4:3/2,
%            ]
%            {(4*(-3+2*x)*(-3+2*y))/9};
%
%
%          \end{axis}
%
%        \end{scope}
%        \draw[thick,fill opacity=1.1,dashed] (a)--(e);
%        \draw[thick,fill opacity=1.1,dashed] (b)--(f);
%        \draw[thick,fill opacity=1.1,dashed] (c)--(g);
%        \draw[thick,fill opacity=1.1,dashed] (d)--(h);
%      \end{tikzpicture}
%      \caption{$h$-unrefinement}
%      \label{subfig:hunref}
%    \end{center}
%  \end{subfigure}
%  \caption{$h$-adaptive operations for a given leaf element. In both cases, the final mesh corresponds to the colored elements (see comment)}	
%  \label{fig:h_refinements}
%\end{figure}



%\begin{figure}
%
%  \newsavebox\foobox
%  \newcommand\slbox[2]{%
%    \FPdiv{\result}{#1}{57.296}% CONVERT deg TO rad
%    \FPtan{\result}{\result}%
%    \slantbox[\result]{#2}%
%  }%
%  \newcommand{\slantbox}[2][30]{%
%    \scalebox{1}[.7]{\mbox{%
%        \sbox{\foobox}{#2}%
%        \hskip\wd\foobox
%        \pdfsave
%        \pdfsetmatrix{1 0 #1 1}%
%        \llap{\usebox{\foobox}}%
%        \pdfrestore
%      }}}
%  \newcommand\rotslant[3]{\rotatebox{#1}{\slbox{#2}{#3}}}
%
%  \usetikzlibrary{positioning,patterns,shapes}
%  \usepgfplotslibrary{external}
%  \usetikzlibrary{decorations.markings,intersections,calc}
%  \usetikzlibrary{decorations.pathreplacing}
%  \usetikzlibrary{fpu}
%
%  \usepgfplotslibrary{colormaps}
%  \usepgfplotslibrary{colorbrewer}
%  \usepgfplotslibrary{patchplots}
%
%  \pgfplotsset{compat=newest}
%
%  \begin{tikzpicture}[scale=1]%,every node/.style={minimum size=1cm},on grid]
%    \begin{axis}[view={40}{70}, ticks=none,hide axis,]
%      \addplot3[patch, patch type=rectangle,
%        patch refines=1, mesh, black, thick,
%        z filter/.code={\def\pgfmathresult{-0.0}} % change that number to translate the "mesh" (beware it may change the color mapping)
%      ] coordinates {
%          (0,0,0) (10,0,0) (10,10,0) (0,10,0)
%        };
%
%      \addplot3[patch, patch type=rectangle,
%        patch refines=1, mesh, black, thick,
%        z filter/.code={\def\pgfmathresult{-0.0}} % change that number to translate the "mesh" (beware it may change the color mapping)
%      ] coordinates {
%          (0,5,0) (0,10,0) (5,10,0) (5,5,0)
%        };
%
%
%      \coordinate (a) at (0,0,0.2);
%      \coordinate (b) at (0,0,1);
%
%      \coordinate (i) at (10,10,0.2);
%      \coordinate (j) at (10,10,1);
%
%      \coordinate (k) at (0,10,0.2);
%      \coordinate (l) at (0,10,1);
%
%      \draw[blue,line width=0.3mm] (0,10,0.2) -- (5,10,1);
%      \draw[blue,line width=0.3mm] (5,10,1) -- (10,10,0.2);
%
%      \draw[blue,line width=0.3mm] (0,10,1) -- (5,10,0.2);
%      \draw[blue,line width=0.3mm] (5,10,0.2) -- (10,10,1);
%
%      \draw[red,line width=0.3mm] (0,10,0.2) -- (2.5,10,1);
%      \draw[red,line width=0.3mm] (2.5,10,1) -- (5,10,0.2);
%
%      \draw[blue,line width=0.3mm] (0,0,0.2) -- (0,5,1);
%      \draw[blue,line width=0.3mm] (0,5,1) -- (0,10,0.2);
%
%      \draw[blue,line width=0.3mm] (0,0,1) -- (0,5,0.2);
%      \draw[blue,line width=0.3mm] (0,5,0.2) -- (0,10,1);
%
%      \draw[red,thick] (0,5,0.2) -- (0,7.5,1);
%      \draw[red,line width=0.3mm] (0,7.5,1) -- (0,10,0.2);
%
%    \end{axis}
%    \draw[thick,fill opacity=1,dashed] (a)--(b);
%    \draw[thick,fill opacity=1,dashed] (i)--(j);
%    \draw[thick,fill opacity=1,dashed] (k)--(l);
%  \end{tikzpicture}
%  \caption{Removable linear shape functions (red).}
%  \label{fig:linearremovable}
%\end{figure}


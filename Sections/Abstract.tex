% !TEX root =  ../unref_general.tex


This work extends an automatic energy-norm $hp$-adaptive strategy based on performing quasi-optimal unrefinements to the case of non-elliptic problems and goal-oriented adaptivity. The proposed approach employs a multi-level hierarchical data structure and alternates global $h$- and $p$-refinements with a coarsening step. Thus, at each unrefinement step, we \rev{eliminate} the basis functions with the lowest contributions to the solution. When solving elliptic problems using energy-norm adaptivity, the \rev{removed} basis functions are those with the lowest contributions to the energy of the solution. For non-elliptic problems or goal-oriented adaptivity, we propose an upper bound of the error representation expressed in terms of an inner product of the specific equation, leading to error indicators that deliver quasi-optimal $hp$-unrefinements. This unrefinement strategy removes unneeded unknowns possibly introduced during the pre-asymptotical regime. In addition, the grids over which we perform the unrefinements are arbitrary, and thus, we can limit their size and associated computational costs. We numerically analyze our algorithm for energy-norm and goal-oriented adaptivity. In particular, we solve two-dimensional ($2$D) Poisson, Helmholtz, convection-dominated equations, and a three-dimensional ($3$D) Helmholtz-like problem. In all cases, we observe \revb{exponential} convergence rates. Our algorithm is robust and straightforward to implement; therefore, it can be easily adapted for industrial applications.




% PREVIOUS 2
% This work extends a novel and automatic energy-norm mesh-refinement strategy, based on a multi-level hierarchical data structure, to non-elliptic problems and goal-oriented adaptivity. The employed $hp$-strategy alternates global $h$- and $p$-refinements with local and quasi-optimal unrefinements. Each unrefinement step marks the basis functions with the lowest contributions to the solution and removes them. For elliptic problems and energy-norm adaptivity, the marked basis functions are those with the lowest contributions to the energy of the solution. For non-elliptic problems and goal-oriented adaptivity, we propose an upper bound of the error representation expressed in terms of an inner product of the specific problem. That leads to error indicators that deliver quasi-optimal unrefinements. We remark that this straightforward unrefinement strategy can remove (correct) some of the undesirable unknowns introduced with the global refinements. In addition to this, the use of rather coarse grids -that are refined and subsequently globally unrefined- avoids, in general, solving expensive fine-grid problems. We numerically analyze our algorithm on two-dimensional ($2$D) Poisson, Helmholtz, and convection-dominated diffusion equations, and in all cases, we observe almost exponential convergence rates. Our algorithm is robust and straightforward to implement, and therefore, it may be easily adapted to industrial applications.






% PREVIOUS 1:

%This work extends a novel automatic energy-norm mesh-refinement strategy based on a multi-level hierarchical data structure to non-elliptic problems and goal-oriented adaptivity. The employed $hp$-strategy alternates global $h$- and $p$-refinements with local and optimal unrefinements. For energy-norm adaptivity, each unrefinement step marks the basis functions with the lowest contributions to the energy of the solution and removes them. For goal-oriented adaptivity and possibly non-elliptic problems, we propose an upper bound of the error representation expressed in terms of an inner product of the problem. That leads to error indicators that deliver optimal unrefinements. We numerically analyze our algorithm on two-dimensional ($2$D) Poisson, Helmholtz, and convection-dominated diffusion equations, and in all cases, we observe exponential convergence rates. Our algorithm is robust and straightforward to implement; therefore, it may be easily adapted to industrial applications.